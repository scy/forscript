\documentclass[ignorenonframetext,ucs]{beamer}

\usepackage[ngerman]{babel}
\usepackage{ucs}
\usepackage[utf8x]{inputenc}
\usepackage[T1]{fontenc}
\usepackage{relsize}
\usepackage{lmodern}

\PreloadUnicodePage{0}

% \usepackage[colorlinks]{hyperref}
\newcommand\refurl[2]{#2\footnote{\url{#1}}}

\usepackage{listings}
\usepackage{graphicx}
% \usepackage[scaled]{luximono}
\definecolor{darkgreen}{rgb}{0,0.5,0}
\lstset{language=,extendedchars=false,basicstyle=\small\ttfamily\color{lightgray},backgroundcolor=\color{black},escapechar=\%}
\newcommand\SH[1]{{\color{green}\$ }{\color{white}#1}}
\def\TYPE{\lstinline[basicstyle=\small\ttfamily\color{darkgreen}]}
\def\cmd{\textsc}
\usetheme[compress]{Berlin}

% \AtBeginSection[]
% {
%   \begin{frame}<beamer>{Übersicht}
%     \tableofcontents[currentsection,currentsubsection]
%   \end{frame}
% }

\title{Design~and~Implementation~of a~Forensic~Documentation~Tool for~Interactive~Command\mbox{-}line~Sessions}
\subtitle{Abschlussvortrag}
\author{Tim Weber}
\date{15. März 2010}
\begin{document}

\begin{frame}\maketitle\end{frame}

% \begin{frame}{Der Vortragende}
% \begin{itemize}
% \item Student der Software- und Internettechnologie
% \item Interessenschwerpunkte: \\ IT-Sicherheit, Netzwerkprotokolle, Datenbanken
% \end{itemize}
% \end{frame}

\section{\cmd{script}}

\subsection{Einsatzgebiet}

\begin{frame}{Protokollierung von Terminalsitzungen}
\begin{itemize}
\item Im Zuge forensischer Ermittlungen
\item Umfassende und detaillierte Aufzeichnung \\ der durchgeführten Tätigkeiten
\item Dokumentation für\begin{itemize}
\item den Ermittler selbst
\item mögliche Amtsnachfolger
\item unabhängige Gutachter o.Ä.\end{itemize}
\end{itemize}
\end{frame}

\subsection{Verwendung von \cmd{script}}

\begin{frame}[fragile=singleslide]{Einsatzbeispiel \cmd{script}}
\begin{example}
\begin{lstlisting}
%\SH{script -t typescript 2> timing}%
Script started, file is typescript
%\SH{dd if=/dev/sda | nc 10.10.1.1 1234}%
...
%\SH{logout}%
Script done, file is typescript
\end{lstlisting}
\end{example}
\end{frame}

\begin{frame}[fragile=singleslide]{Was tut \cmd{script}?}
\begin{example}
\begin{lstlisting}
%\SH{script}%
Script started, file is typescript
%\SH{ps faux}%
...
3704 ?     S  /usr/bin/x-terminal-emulator
3705 pts/4 Ss  \_ -bash
3843 pts/4 S+      \_ script
3844 pts/4 S+          \_ script
3845 pts/7 Ss              \_ bash -i
3878 pts/7 R+                  \_ ps faux
\end{lstlisting}
\end{example}
\begin{center}
Terminal-Emulator $\Leftrightarrow$ \emph{pts/4} $\Leftrightarrow$ \cmd{script} $\Leftrightarrow$ \emph{pts/7} $\Leftrightarrow$ \cmd{bash}
\end{center}
\end{frame}

\subsection{Ausgabeformat}

\begin{frame}[fragile=singleslide]{Typescript}
\begin{itemize}
\item 1:1-Ausgabe der als Kindprozess laufenden Software
\item Inklusive Steuerzeichen etc.
\end{itemize}
\begin{example}
\begin{verbatim}
^[[1;32mscy@ ~ $^[[0m echo test^M
test^M
\end{verbatim}
\end{example}
\end{frame}

\begin{frame}[fragile=singleslide]{Timingdaten}
\begin{itemize}
\item ASCII-Daten
\item Zu wartende Zeit in Sekunden; Anzahl auszugebender Bytes
\end{itemize}
\begin{example}
\begin{verbatim}
0.163965 117
0.038713 1
9.803904 1
0.175984 1
0.023397 2
…
\end{verbatim}
\end{example}
\end{frame}

\begin{frame}{Schwächen}
\begin{itemize}
\item Nur Bildschirmausgabe wird protokolliert, \\ Eingaben überhaupt nicht\begin{itemize}
\item Tab-Completion? \texttt{\char`\^C}/\texttt{\char`\^D}/\texttt{\char`\^M}?\end{itemize}
\item Timinginformationen werden in separater Datei gespeichert
\item Keine Informationen über die Laufzeitumgebung\begin{itemize}
\item Zeichenkodierung
\item Umgebungsvariablen
\item Fenstergröße
\item …\end{itemize}
\end{itemize}
\end{frame}

\subsection{Die Neuentwicklung: \cmd{forscript}}

\begin{frame}{Ziele der Bachelorarbeit}
\begin{itemize}
\item Ausführliche Dokumentation des Verhaltens von \cmd{script}, Aufzeigen seiner Schwächen
\item Gestalten eines neuen Dateiformates, \\ das forensischen Ansprüchen genügt
\item Entwickeln einer Software, die in diesem Format protokolliert
\item Dokumentieren dieser Software nach den Prinzipien \\ des \emph{literate programming}
\end{itemize}
\end{frame}

\section{Das neue Datenformat}

\subsection{Grundlegendes}

\begin{frame}{Bestandteile einer \cmd{forscript}-Datei}
Ein einziger chronologisch fortlaufender Datenstrom, enthält:
\begin{itemize}
\item Ausgaben der Clientsoftware
\item Eingaben des Benutzers
\item Metainformationen\begin{itemize}
\item die Datei strukturierend
\item den Datenstrom ergänzend\end{itemize}
\end{itemize}
\end{frame}

\subsection{Datentypen}

\begin{frame}{Ausgaben der Clientsoftware}
\begin{itemize}
\item Identisch zu \cmd{script}
\item Steuerzeichen werden 1:1 übernommen
\item Für maximale Authentizität werden \\ keinerlei Änderungen vorgenommen\begin{itemize}
\item Keine Umkodierung in Unicode etc.
\item Ausnahme: Escaping von Bytes mit Sonderfunktion
\item Rekonstruktion/Wiedergabe: Aufgabe der lesenden Software, \\ die ursprüngliche Situation zu erkennen, \\ z.B. anhand Umgebungsvariablen\end{itemize}
\end{itemize}
\end{frame}

\begin{frame}{Eingaben des Benutzers}
\begin{itemize}
\item Durch vorangestelltes \emph{shift out, shift out} (\texttt{0x0e 0x0e}) signalisiert
\item Unveränderte Eingabebytes (Ausnahme: Escaping)\begin{itemize}
\item Druckbare Zeichen
\item Cursorbewegungen
\item Tastenkombinationen
\item Sondertasten
\item …\end{itemize}
\item Abschluss durch \emph{shift in} (\texttt{0x0f})
\end{itemize}
\end{frame}

\begin{frame}{Metainformationen}
\begin{itemize}
\item Durch vorangestelltes \emph{shift out} (\texttt{0x0e 0x0e}) und \\ ein Typ-Byte signalisiert
\item Festgelegte Typ-Bytes, aber in künftigen Versionen erweiterbar\begin{itemize}
\item Dateiformat-Version
\item Anfang einer neuen Sitzung
\item Umgebungsvariablen
\item Fenstergröße
\item Sprache und Zeichensatz/Encoding
\item Timinginformationen
\item …\end{itemize}
\item Terminiert durch \emph{shift in} (\texttt{0x0f})
\end{itemize}
\end{frame}

\subsection{Eigenschaften}

\begin{frame}{Escaping}
\begin{itemize}
\item \emph{shift out} (\texttt{0x0e}) und \emph{shift in} (\texttt{0x0f}) werden durch vorangestelltes \emph{data link escape} (\texttt{0x10}) escaped
\item \emph{data link escape} muss folglich auch escaped werden: \\ durch sich selbst
\begin{example}\begin{center}
\texttt{0x4e 0x0f 0x00 0x61 0x74 0x10}\\
$\Downarrow$\\
\texttt{0x4e 0x10 0x0f 0x00 0x61 0x74 0x10 0x10}
\end{center}\end{example}
\end{itemize}
\end{frame}

\begin{frame}{Zeitmessung}
\begin{itemize}
\item Hauptschleife: Warten auf\begin{enumerate}
\item Benutzereingaben
\item Clientausgaben
\item Signale (Fenstergröße verändert, Client beendet)\end{enumerate}
\item Vor Bearbeiten eines jedes Ereignisses wird Zeitdifferenz \\ in Ausgabedatei geschrieben
\end{itemize}
\end{frame}

\section{\cmd{forscript}}

\subsection{Implementierung}

\begin{frame}{Standards}
\begin{itemize}
\item \cmd{forscript} hält sich strikt an Unix-übergreifende Standards\begin{itemize}
\item C 99
\item POSIX.1-2001
\item System V r4
\item keine BSD-Funktionen\end{itemize}
\item Lauffähig auf Linux und NetBSD, \\ möglicherweise auch auf anderen
\item Auf OS~X bislang nicht lauffähig, \\ da dort \texttt{clock\_gettime()} fehlt\begin{itemize}
\item Kann aber emuliert oder ersetzt werden\end{itemize}
\end{itemize}
\end{frame}

\begin{frame}{Neuerungen}
\begin{itemize}
\item \cmd{script} etwa 1997 geschrieben, \\ neuere Library-Funktionen existieren:\begin{itemize}
\item \texttt{select()} statt mehreren Prozessen
\item \texttt{posix\_openpt()} statt race-behaftetem \\ Durchprobieren von PTYs
\item \texttt{clock\_gettime()} mit monotonem Zeitverlauf\end{itemize}
\item \cmd{forscript} ignoriert \emph{SIGSTOP} des Childs
\item \texttt{-a} mit Timing unterstützt
\item Nur 1× forken $\Rightarrow$ vereinfachter Code
\end{itemize}
\end{frame}

\end{document}
